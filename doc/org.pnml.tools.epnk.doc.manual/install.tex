%%%%%%%%%%%%%%%%%%%%%%%%%%%%%%%%%%%%%%%%%%%%%%%%%%%%%%%%%%%%%%%%%%%%%%%%%%%%%%%
%% Installation                                                            %%
%%%%%%%%%%%%%%%%%%%%%%%%%%%%%%%%%%%%%%%%%%%%%%%%%%%%%%%%%%%%%%%%%%%%%%%%%%%%%%%

\chapter{Installation}
\label{chap:install}


This chapter discusses the installation of the ePNK (version 1.0.0). Readers who
are interested in getting an idea of what the ePNK is and who do not want to
work with the PNK right away can skip this chapter.

\section{Prerequisites}
\urldef{\JRE16URL}{\url}{http://wiki.eclipse.org/FAQ_How_do_I_run_Eclipse%3F
}
In order to install the ePNK, you need to have Java~1.6 (or higher)
and Eclipse~3.7 (Indigo) or Eclipse~4.2 (Juno) installed on your computer.
In this version of the manual, we discuss the installation of the ePNK 
version 1.0.0 only. For installing other versions of the ePNK or for
installing it on other versions of Eclipse, you might find information
on the \emph{ePNK installation page}\footnote
  {\url{http://www2.imm.dtu.dk/~ekki/projects/ePNK/install-details.html}}%
.

For the installation of Java, please refer to \url{http://www.java.com/}.


If you are new to Eclipse, it is recommended that you install the \emph{Eclipse
Classic} version.%
  \index{Eclipse!Installation|DEF}
Download this Eclipse version for your operating
system from \url{http://www.eclipse.org/downloads/} and extract the
downloaded file to some directory; after the extraction,  you will find a folder
named ``eclipse'' in this directory, and in this folder, you will find an
executable file also called ``eclipse'' (e.\,g.\ ``eclipse.exe'' on the Windows platform). 
Executing this file will start Eclipse.

If you are new to Eclipse, you can get a quick overview of the
Eclipse Integrated Development Environment (IDE) at% 
  \index{Eclipse!IDE}%
  \index{IDE|see{Eclipse}}
\url{http://www.vogella.de/articles/Eclipse/article.html}. Once you have
installed and started Eclipse, you will find much more information on
Eclipse in the ``Workbench User Guide" in the Eclipse help: You can open it
via the ``Help'' menu in the Eclipse toolbar under ``Help Contents''.

\section{Installing the ePNK in Eclipse}
\index{ePNK!Installation|(DEF}
Once you have installed Eclipse, you can install the ePNK from
the Eclipse workbench. To this end, the ePNK is made available
via an \emph{Eclipse update site}%
  \index{ePNK!Update site|DEF}%
  \index{Update site|see{ePNK}}%
: \url{http://www2.imm.dtu.dk/~ekki/projects/ePNK/indigo/update/}


In order to install the ePNK from there to your Eclipse installation, you should
proceed as follows (after you have started it and selected a workspace):
\begin{enumerate}
\item In the Eclipse toolbar, select 
      ``Help'' $\rightarrow$ ``Install New Software...'', which will open
      an install dialog.

\item  In the install dialog, press the ``Add...'' button to add
       a new update site. In the ``Add Site'' dialog, enter some name 
       (e.\,g.\ ``ePNK Update Site'') and the URL
       
       \url{http://www2.imm.dtu.dk/~ekki/projects/ePNK/indigo/update/}
       
       as location, and then press okay.
       
\item Now, select the newly created ePNK update site in the still open
      install dialog. After some time, some ePNK items should pop up in the
      dialog. From there, you can select the features of the ePNK
      you want.
      
      For working with this manual, you should at least select the following
      features from the category ``ePNK Features'':
      \begin{itemize}
      \item ePNK Basic Extensions~1.0.0
      \item ePNK Core~1.0.0
      \item ePNK HLPNGs~1.0.0
      \item ePNK Tutorial~1.0.0 
      \end{itemize}
      
      If you intend to import high-level nets from other tools than the
      ePNK, it is recommended that you also install the feature 
      \begin{itemize}
      \item ePNK: HLPNG Label Serialisation (experimental)~0.2.0 
      \end{itemize}
      from category ``ePNK Experimental Projects''.
      
      If you want to simulate high-level nets, you should also select
      the feature
      \begin{itemize}
      \item ePNK: HLPNG Simulator~0.1.1  
      \end{itemize}
      from category "HLPNG Simulator".
      
      You will not need the features from the ``ECNO Projects'' category,
      which are a project in their own right (see \cite{Kin12b} for more
      information on the ECNO project). Since they are based on the ePNK, they
      are deployed from the same update site.

\item After you have selected the features you want, make sure that
      the box ``Contact all update sites during
      install to find required software'' is checked; this will guarantee that
      all additional features from Eclipse that the ePNK requires will be
      automatically installed together with the ePNK features (EMF, GMF, Xtext,
      etc.).
      
      Then press press okay.
      
\item Follow through the installation process (don't forget to
      accept the license agreement).
 
      \begin{quote} 
      {\bf Note}:  If you get an error of the kind
      \begin{quote}
      {\tt Cannot complete the install because one\\
           or more required items could not be found.\\
            ... }
      \end{quote}
      you probably forgot to check the box ``Contact all update sites during
      install to find required software'' or have selected a wrong combination
      of features. In that case, go back and select the right combination as
      explained above.
      \end{quote}

\item Then, the selected features of the ePNK and all other required features
      will be installed; it is a good idea to restart Eclipse after that
      (Eclipse will ask you to do that anyway).
\end{enumerate}

In case you intend to develop new functions and, in particular, new Petri net
types for the ePNK, you might want to install the tools necessary for that
purpose already now -- while at it. You need to install the
``EMF Modeling Framework SDK'' and the ``Ecore Tools SDK'' from the standard
Eclipse update site. The details are described in Sect.~\ref{subsec:installingEcoreTools}.%
  \index{ePNK!Installation|)}