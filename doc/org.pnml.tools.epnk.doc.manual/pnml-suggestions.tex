%%%%%%%%%%%%%%%%%%%%%%%%%%%%%%%%%%%%%%%%%%%%%%%%%%%%%%%%%%%%%%%%%%%%%%%%%%%%%%%
%% PNML Suggestions                                                          %%
%%%%%%%%%%%%%%%%%%%%%%%%%%%%%%%%%%%%%%%%%%%%%%%%%%%%%%%%%%%%%%%%%%%%%%%%%%%%%%%

\chapter{PNML: Observations and suggestions}
\label{chap:pnml-suggestions}

During the work on the ePNK, several issues came up concerning PNML.
In this chapter, these issues will be summarized and some suggestions
on how to improve future versions of PNML will be made. 


\section{Toolspecific extensions: Type attribute}

The toolspecific extensions should probably be equipped with an
optional attribute type: The rational is that the same tool
could have different toolspecific extensions. With the additional
attribute, we would not force the tool to mix all its extensions
in a single class (or define the type in an extension of the
tool name, which would be also a mess).

\section{Relaxing requirement for ids everywhere}  
PNML requires an id for every element in the document. In the ePNK,
I have made the ids optional, when they are not needed for referring
to an element. And id is only required, if there is a reference to such an
element (and actually the document can still be serialized using XPath
expressions without ids at all, which however is not compatible with PNML);

For future versions of PNML, it should be considered to make ids optional in
case elements cannot be referenced at all or when elements are not referenced
from other elements in the net.
  
\section{Naming in PNML standard}
The feature variableDecl in the UML model of Variable in the
terms package is later mapped to refvariable in XML. This
could actually be the same name in both cases (the ePNK model used
``refvariable'' in the model). 
%
For the definition of PNML as an exchange format, this does not make any 
difference -- but, it is smoother conceptually, technically, and from a
readability point of view, since we do not need an extra mapping.

In the terms package, there is a class BuiltInConst. Later (e.\,g.\ in
packages dots, booleans, integers) it is referred to as BuiltInConstants.
this needs to be aligned. In ePNK, BuiltInConst is use in all cases.
  
In package booleans, Equality and Inequality are directly derived from
Operators. Though, it does not make a difference for PNML as a transfer
format, this is probably an error. In the ePNK model, Equality and Inequality
are derived from BuiltInOperators, which makes explicit that these are
built-in operators in the model.  

In ISO/IEC 15909-2, Table~8 (p.~35) the declaration feature of a variable is
mapped to the attribute variabledecl; in the RELAX NG grammar, however,
this is represented as "refvariable" (Annex B.2.1, p.~54) and also in the
example of Symmetric Nets (AnnexC, e.\,g.\ p.~93). This is probably a type
and should be aligned! In the ePNK ``refvariable'' was used.  
  
Also in Table~8, the Integer::Addition class is mapped to "add". But,
in the RELAX/NG grammars this is called ``addition''. 
The ePNK uses ``addition'', and ``subtraction'' to be consistent with
the RELAX/NG grammar. But ``addition'' and ``subtraction'' is not
consistent with the naming of ``div'', ``mod'', and ``mult''. This could
be aligned.
   
\section{Mapping of models to XML}   

The sorts ProductSort and MultisetSort (and their children) are not
mapped to XML in the same style as Terms (there is no element for
the feature (elementSort or basis), but just an element for the
sort itself. That is a bit awkward and we might want to align this.
  
The same applies for NamedSorts.
  
In order to map these features of PNML (resp. HLPNG) to XML, the
ePNK introduced the concept of \emph{standard features}.  

\section{Explict result type for NamedOperators}
A NamedOperator should have an explicit sort for its return value. Right
now, we rely on that the definition of NamedOperators is not cyclic --
which is explicitly required. But, if this is not checked,
other calculations might run into infinite loops, if not done
properly. This could be done more easily if the return type
of an operator was made explicit.
   
\section{Opposite references for source and target}
Maybe, the source and target references of arcs in the
PNML Core Model should have opposites, so that the arcs can
be accessed from the  nodes more easily (they should be transient
though, since they do not need to be serialized). I will
make this change in the ePNK anyway (technically, this
has nothing to do with ISO/IEC 15909-2, since it does not
mandate the PNML Core Model, but only the XML format -- but,
it might be relevant for Part~3).   

\section{Name labels for arcs}   
I still do not see any reason, why arcs of a Petri net should
have a name.

\section{Remove the possibility of net labels}   
We should consider to remove the labels that are directly contained
in a net (and not on page) from the PNML Core Model; this does
not make much sense conceptually and makes things more difficult
to implement (see Sect.~\ref{subsec:user-netlabels}).

\section{Attributes with default values}    
The RELAX/NG grammars of ISO/IEC~15909-2 require that all attributes of
all elements must of must always be there. Many other formats allow omitting
attributes with default values. We should discuss, whether ISO/IEC~15909 should
also allow to omit attributes that have a default value (and define what the
default values are). 

In the current implementation, the ePNK serializes all attributes as defined
in ISO/IEC~15909-2, but it is no problem changing that (and for all attributes,
the ePNK defines some default values in the models -- since, strangely enough,
this was the only way to enforce their serialisation in the EMF technology).

\section{Recursive sort and operator definitions}
Up to now, PNML does not allow defining recursive sorts and has
limited expressibility for defining recursive functions. There
could be more constructs for that; but this would require to
have a more careful definition of this (right now the semantics
of sorts is to flatten them, which would not work for recursive
types anymore).

We also might want unions for defining recursive sorts in an easy
way. But, this would impose a great burden on all implementations
of PNML even if they do not use it -- therefore, we should be
very careful with that (it could be part of a even more general
kind of HLPNGs).

\section{Names for ArbitrayDeclarations}

As mentioned in Sect.~\ref{subsec:user-netlabels} already, {\tt Unparsed}
declarations of ISO/IEC do not have a name attribute. Since, any
declaration is a symbol definition, of some kind, it would be
reasonable, if this would be aligned. Therefore, {\tt Unparsed}
should have a name attribute of type String.


\section{Incorrect types in the HLPNG models}   
\label{ISO-IEC15909-problem:typing} 
In ISO/IEC 15909-2, some classes in the packages FiniteIntRanges, Strings, and
Lists, have attributes that are technically incorrect. They are on the
wrong meta-level: they refer to a PNML/HLPNG type instead of a data type
in UML. This should be changed

Here is a list of the required changes:
\begin{itemize}
\item {\tt FiniteIntRange}: {\tt start} and {\tt end} should refer to {\tt int}
      ({\tt EInt})
       
\item {\tt FiniteIntRangeConstant}: {\tt value} should refer to {\tt int}
      ({\tt EInt})
      
\item {\tt Substring}: {\tt start} and {\tt length} should refer to
      {\tt NonNegativeInteger} (is defined as a separate basic datatype in the
      ePNK) 

\item {\tt Sublist}: {\tt start} and {\tt length}  should refer to {\tt NonNegativeInteger}
       
\item {\tt MemberAtIndex}: {\tt index} should refer to {\tt NonNegativeInteger}
      (this attribute is capitalized in the wrong way in Fig. 19, which should
      also be changed) 
\end{itemize}

Moreover, the type {\tt PrimitiveType::Integer} that is used in the {\tt value}
feature of {\tt NumberConstant} in the Integers package is not defined within
ISO/IEC~15909-2. This should be changed (in the ePNK the type  {\tt EInt} is
used).

\section{Reference to Sort in FiniteIntRangeConstants}  
The {\tt range} feature of {\tt FiniteIntRangeConstant} refers to 
{\tt FiniteIntRange}. This makes the concrete syntax for
these constants (and the implementation of a parser for this concrete syntax)
very inconvenient. It would be much easier, if this {\tt range} feature referred
to {\tt Sort} in general, and there was an additional constraint that
this sort is (or refers to) a {\tt FiniteIntRange}.

\section{Omitting redundant opposit references}
In the package for Partitions, there are the features {\tt refpartition} and
{\tt refpartitionelement}, which are redandant and are not serialised. It
would be better to delete them form these models (in the API generatded by
EMF, the respective elements can be accessed via the eContainer, if necessary).
    
\section{Useless operator: Append}    
The {\tt Append} operator in package Strings, does not make any sense, if
there is no data type for characters. So, either a data type for characters
should be introduced, or {\tt Append} should be removed. 

{\bf Note:} In the ePNK, the  {\tt Append} it is included for compatibility
reasons; but terms using it, will never be typed correctly and cannot be
resolved. If this operator is used, there will be validation errors, but the
respective nets can be serialised.
    
\section{Incorrect constraints for List Append}    
The constraint for the {\tt Append} operator in Lists is incorrect (and
not complete): It requires both arguments to be lists; but if this
was meant seriously, the {\tt Append} operator would not be different
from the list concatenation. One of the arguments of the {\tt Append}
should be of the basis sort of the list (representing a single
element that is appended to the list).  In the ePNK, the first
argument is the of the basis sort for now. 
    
\section{Implicit and constant parameteres}    
The operators {\tt Substring} of package Strings and {\tt MemberAtIndex} and
{\tt Sublist} of package Lists have constant implicit parameters for
the start, length, resp. index. These should rather be subterms
of the respective types (expressed as constraints); maybe this was the intention
with the wrong types (see~\ref{ISO-IEC15909-problem:typing}), but the
RELAX/NG grammar clearly states that these are not subterms.

In the ePNK, this is implemented according to ISO/IEC 15909-2, but in a
future version, we will (optionally) allow resp.\ subterms; having only
constants here, is not too useful.
    
\section{Mismatch between tables and grammars} 
In the mapping of the PNML core model and type specific models to XML in
Clause~7 (Table~8) of ISO/IEC~15909-2, the {\tt FiniteIntRangeConstant} has a
PNML attribute {\tt range} which is an IDREF.
This does not correspond to the RELAX/NG grammar and the UML model. This
should rather be a (default) sub-element of type Sort (this is what was
implemented in the ePNK and is suggested by the RELAX/NG gammar).
In ISO/IEC 15909-2, the entry "range:IDREF" in Table~8 for
{\tt FiniteIntRangeConstant} should be deleted.
  
